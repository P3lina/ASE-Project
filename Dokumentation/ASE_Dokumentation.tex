% ------------------------------------------------------------
% LaTeX Template für die DHBW zum Schnellstart!
% Original: https://github.wdf.sap.corp/vtgermany/LaTeX-Template-DHBW
% ------------------------------------------------------------
% ---- Präambel mit Angaben zum Dokument
\input{Inhalt/00_Latex/praeambel}

% ---- Elektronische Version oder Gedruckte Version?
% ---- Unterschied: Die elektronische Version enthält keinen Platzhalter für die Unterschrift
\usepackage{ifthen}
\newboolean{e-Abgabe}
\setboolean{e-Abgabe}{false}    % false=gedruckte Fassung

% ---- Persönlichen Daten:
\newcommand{\titel}{Programmentwurf}
\newcommand{\titelheader}{Dart Counter}
\newcommand{\arbeit}{Dart Counter}
\newcommand{\studiengang}{Informatik}
\newcommand{\studienjahr}{2023}
\newcommand{\autor}{Robin Purschwitz}
\newcommand{\autorReverse}{Purschwitz, Robin}
\newcommand{\verfassungsort}{Karlsruhe}
\newcommand{\matrikelnr}{2415691}
\newcommand{\kurs}{TINF20B2}
\newcommand{\bearbeitungsmonat}{Mai 2023}
\newcommand{\abgabe}{25. Mai 2023}
\newcommand{\bearbeitungszeitraum}{12.12.2022 - 25.05.2023}
\newcommand{\betreuerDhbw}{Dr. Lars Briem}

\input{Inhalt/00_Latex/kopfundFusszeile}

% ---- Hilfreiches
\newcommand{\zB}{z.\,B. }   % "z.B." mit kleinem Leeraum dazwischen (ohne wäre nicht korrekt)
\newcommand{\dash}{d.\,h. }

\newcommand{\code}[1]{\texttt{#1}} % Ist einfacher zu schreiben als ständig \texttt und erlaubt
                                   % Änderungen im Nachhinein, wenn man z.B. Inline-Code anders stylen möchte.

% ---- Silbentrennung (falls LaTeX defaults falsch / nicht gewünscht sind)
\hyphenation{HANA}         % anstatt HA-NA
\hyphenation{Graph-Script} % anstatt GraphS-cript

% ---- Beginn des Dokuments
\begin{document}
\setlength{\parindent}{0pt}              % Keine Paragraphen Einrückung.
% Dafür haben wir den Abstand zwischen den Paragraphen.
\setcounter{secnumdepth}{2}              % Nummerierungstiefe fürs Inhaltsverzeichnis
\setcounter{tocdepth}{1}                 % Tiefe des Inhaltsverzeichnisses. Ggf. so anpassen,
% dass das Verzeichnis auf eine Seite passt.
\sffamily                                % Serifenlose Schrift verwenden.

% ---- Vorspann
% ------ Titelseite
\singlespacing
\include{Inhalt/01_Standard/titelseite}  % Titelseite
\newcounter{savepage}
\pagenumbering{Roman}                    % Römische Seitenzahlen
\onehalfspacing

% ------ Inhaltsverzeichnis
\singlespacing
\tableofcontents

% ------ Verzeichnisse
\renewcommand*{\chapterpagestyle}{plain}
\pagestyle{plain}
%\include{Inhalt/03_Verzeichnisse/formelgroessen}
\include{Inhalt/03_Verzeichnisse/abkuerzungen}
\listoffigures                          % Erzeugen des Abbildungsverzeichnisses 
\listoftables                           % Erzeugen des Tabellenverzeichnisses
\renewcommand{\lstlistlistingname}{Quellcodeverzeichnis}
\lstlistoflistings                      % Erzeugen des Listenverzeichnisses
\setcounter{savepage}{\value{page}}


% ---- Inhalt der Arbeit
\cleardoublepage
\pagenumbering{arabic}                  % Arabische Seitenzahlen für den Hauptteil
\setlength{\parskip}{0.5\baselineskip}  % Abstand zwischen Absätzen
\rmfamily
\renewcommand*{\chapterpagestyle}{scrheadings}
\pagestyle{scrheadings}
\onehalfspacing
\chapter{Einführung}
\section{Übersicht über die Applikation}
Dart ist ein beliebtes Geschicklichkeitsspiel, das sowohl als professioneller Sport als auch als geselliges Freizeitspiel gespielt wird. Ziel des Spiels ist es, Punkte zu sammeln, indem man Pfeile (\textit{Darts}) auf ein kreisförmiges Dartboard wirft. Ein Standard-Dartboard ist in 20 nummerierte Sektionen unterteilt, wobei jeder Bereich unterschiedliche Punktzahlen vergibt. Darüber hinaus gibt es Doppel (Double) und Dreifach (Triple Felder), die den Wert der getroffenen Sektion verdoppeln bzw. verdreifachen. Das Feld dreifache 20 (\textit{triple 20}) gibt die meisten Punkte. Es gibt verschiedene Spielvarianten, wobei 501 und 301 die bekanntesten sind. Bei diesen Varianten beginnen die Spieler mit einer festgelegten Punktzahl und müssen versuchen, ihre Punktzahl exakt auf Null zu reduzieren. Das Spiel erfordert Präzision, gute Hand-Auge-Koordination und strategisches Denken. Professionelle Dart-Turniere werden weltweit ausgetragen und ziehen Tausende von Zuschauern an. Der Dart-Counter ist eine Anwendung, die es einer Gruppe aus Spielern ermöglicht dieses Spiel zu spielen. Zwar müssen die Spieler die Darts immer noch händisch auf eine Dartscheibe werfen, jedoch können die Spieler so ihre Punkte zählen und wissen, wie viele Punkte ihnen noch zu einem Checkout fehlen. Darüber hinaus können die Spieler einen ständigen Überblick über ihren aktuellen Average (Durchschnitt) behalten und am Ende eines Legs (\zB einer Runde 501) ihre Checkout-Quote (Trefferquote) sehen. Am Ende eines Gesamten Matches, welches aus mehreren Sets bestehen, welche wiederum aus mehreren Legs bestehen, haben die Spieler die Möglichkeit, ihren Spielverlauf in einer Textdatei zu speichern.
\section{Wie startet man die Applikation}
Zum starten der Applikation wird eine Java-Laufzeitumgebung und ein JDK (Version 19) benötigt. Zusätzlich wird eine \acf{IDE} zum Ausführen der Anwendung benötigt. Mit dieser IDE kann auch die Anwendung gebaut werden, um das Ausführen ohne IDE zu ermöglichen. Gestartet werden muss die Klasse '\textit{DartCounterV2/src/main/java/de/p3lina/Main.java}'. Alle Interaktionen mit der Anwendung finden dann in dem Terminal der IDE statt, mit welcher ausschlieslich mit der Tastatur interagiert werden kann. Um ein eigentliches Match zu spielen, werden zuerst Spielinformationen, wie \zB Anzahl der Spieler, Spielernamen, Startpunktanzahl etc. benötigt. Diese können, wie bereits erwähnt mit der Tastatur spezifiziert werden.
\section{Wie testet man die Applikation}
Das Testen der Applikation erfordert die Installation von Maven. Zum Testen können die Test-Klassen, welche sich in jedem Modul (application, domain etc.) unter '\textit{src/main/test}' befinden, ausgeführt werden. Die Test-Klassen können mit der IDE ausgeführt werden. Um nicht alle Tests einzeln ausführen zu müssen, besteht auch die Möglichkeit mit dem Terminal in das Wurzelverzeichnis zu navigieren und den Befehl '\textit{mvn test}' auszuführen.
\chapter{Clean Architecture}
In diesem Kapitel steht die Clean Architecture und deren zentrale Aspekte im Fokus. Zuerst erfolgt eine Analyse der Dependency Rule, einer Schlüsselregel der Clean Architecture. Untersucht werden dabei die Auswirkungen dieser Regel auf die Softwarearchitektur, ergänzt durch positive und negative Anwendungsbeispiele.

Im Anschluss daran richtet sich der Fokus auf die Struktur der Clean Architecture, wobei die einzelnen Schichten detailliert analysiert werden. Besondere Aufmerksamkeit gilt dabei den Schichten "Domain" und "Application", deren Rolle und Bedeutung innerhalb der Clean Architecture ausführlich diskutiert werden. Diese Analysen ermöglichen ein tiefgreifendes Verständnis der Clean Architecture und ihrer praktischen Anwendung.
\section{Was ist Clean Architecture}
Die \textit{Clean Architecture}, auch bekannt als die \textit{Onion Architecture}, ist ein Software-Entwurfsprinzip, das von Robert C. Martin, entwickelt wurde. Sie zielt darauf ab, eine klare und getrennte Struktur in Software-Systemen zu schaffen, um Wartbarkeit, Testbarkeit und Flexibilität zu verbessern.

Die Clean Architecture teilt eine Anwendung in konzentrische Schichten auf, wobei jede Schicht bestimmte Arten von Aufgaben erfüllt und klar definierte Abhängigkeiten aufweist. Die Schichten, von innen nach außen, sind in der Regel wie folgt:

\begin{figure}[ht]
    \includegraphics[width=1\textwidth]{Bilder/clean-architecture.jpeg}
    \caption{Clean Architecture Schichten}
    \label{fig:clean-architecture}
\end{figure}
\newpage
\textbf{Domain-Schicht}: Dies ist die innere Schicht, die die Geschäftslogik und die Geschäftsregeln einer Anwendung enthält. Sie hat keine Abhängigkeiten von den äußeren Schichten und repräsentiert die fundamentalen Konzepte der Anwendung, unabhängig von spezifischen technologischen Details.\\

\textbf{Anwendungs-Schicht}: Diese Schicht enthält spezifische Geschäftslogik, die sich auf bestimmte Anwendungsfälle bezieht. Sie ist von der Domain-Schicht abhängig und kann mit ihr interagieren, aber sie kennt keine Details über äußere Schichten.\\

\textbf{Adapter-Schicht}: Diese Schicht übersetzt Daten zwischen den Formaten, die für die inneren Schichten und die äußeren Schichten geeignet sind. Sie könnte beispielsweise Datenbankcode, Benutzeroberflächen-Code oder sogar Code für externe Dienste enthalten.\\

\textbf{Plugin-Schicht}: Dies ist die äußerste Schicht, die spezifische Technologien wie Datenbanken, Webserver oder Frameworks umfasst. Sie interagiert mit den inneren Schichten durch Ports und Adapter.
\newpage
Das Hauptprinzip der Clean Architecture ist die Regel der Abhängigkeitsrichtung: Abhängigkeiten sollten immer von äußeren Schichten zu inneren Schichten gerichtet sein. Dies bedeutet, dass der Code in den inneren Schichten unabhängig von spezifischen Frameworks, Datenbanken oder anderen Technologien ist, was ihn einfacher zu testen und zu warten macht.\\

Zusätzlich wird durch die klare Trennung der Verantwortlichkeiten die Einhaltung des \acf{SRP} und des \acf{OCP} aus den SOLID-Prinzipien erleichtert. Es ermöglicht auch eine bessere Modularität und Austauschbarkeit der Komponenten, da Änderungen in einer Schicht sich nicht auf die anderen Schichten auswirken sollten.

Zusammenfassend lässt sich sagen, dass die Clean Architecture ein Ansatz ist, der darauf abzielt, die Unordnung und Komplexität in Softwareprojekten zu reduzieren, indem klare Grenzen und Regeln für die Struktur und Organisation des Codes vorgegeben werden. Sie ermöglicht es Entwicklern, Systeme zu erstellen, die widerstandsfähig gegenüber technologischen Änderungen sind und die sich im Laufe der Zeit leicht anpassen und erweitern lassen.
\section{Analyse der Dependency Rule}
d
\section{Positiv-Beispiel: Dependency Rule}
\section{Negativ-Beispiel: Dependency Rule}
\section{Analyse der Schichten}
\subsection{Schicht: Domain}
\subsection{Schicht: Application}
\chapter{SOLID}
SOLID ist ein Akronym, das fünf grundlegende Prinzipien des objektorientierten Designs zusammenfasst. Diese Prinzipien dienen dazu, die Struktur und Flexibilität von Software zu verbessern und die Codequalität zu erhöhen.
\begin{enumerate}
    \item Single Responsibility Principle (\textbf{SRP}): Eine Klasse sollte nur eine einzige Verantwortlichkeit haben und für eine spezifische Aufgabe zuständig sein.

    \item Open-Closed Principle (\textbf{OCP}): Klassen sollten offen für Erweiterungen, aber geschlossen für Modifikationen sein, indem sie neue Funktionen hinzufügen, ohne den bestehenden Code zu ändern.
    
    \item Liskov Substitution Principle (\textbf{LSP}): Subtypen sollten sich genauso verhalten wie ihre Basistypen, um eine nahtlose Austauschbarkeit zu gewährleisten.
    
    \item Interface Segregation Principle (\textbf{ISP}): Schnittstellen sollten spezifisch auf die Bedürfnisse der Clients zugeschnitten sein, sodass sie nur von den Methoden abhängig sind, die sie tatsächlich benötigen.
    
    \item Dependency Inversion Principle (\textbf{DIP}): Abhängigkeiten sollten auf abstrakte Konzepte oder Schnittstellen, nicht auf konkrete Implementierungen, ausgerichtet sein, um die Flexibilität und Austauschbarkeit von Komponenten zu fördern.
\end{enumerate}
\section{Analyse Single-Responsibility-Principle (SRP)}
Das Single Responsibility Principle (SRP) besagt, dass eine Klasse nur eine einzige Verantwortlichkeit haben sollte. Sie sollte für eine spezifische Aufgabe oder Funktion zuständig sein. Dadurch wird sichergestellt, dass die Klasse nur für einen bestimmten Aspekt der Funktionalität verantwortlich ist und sich nicht mit mehreren unterschiedlichen Aufgaben befasst. Das SRP ermöglicht eine bessere Lesbarkeit, Wartbarkeit und Testbarkeit des Codes, da jeder Verantwortungsbereich in einer separaten Klasse organisiert ist. Durch die Einhaltung des SRP können Änderungen oder Erweiterungen in einem Bereich vorgenommen werden, ohne dass andere Bereiche der Klasse betroffen sind.
\subsection{Positiv Beispiel}
\begin{figure}[ht]
    \centering
    \includegraphics[width=0.8\textwidth]{Bilder/Leg.png}
    \caption{Leg Klasse}
    \label{fig:leg-uml}
\end{figure}
Diese Klasse hält das SRP ein, da sie nur die Aufgabe hat, Informationen über ein Leg eines Dart Spiels zu speichern. Zur Initialisierung des Legs werden 2 Informationen benötigt. Die Spieler, welche am Leg teilnehmen und die Anzahl der zu erreichenden Punkte. Während der Initialisierung wird das Attribut \textit{playerscore} deklariert und mit einer \textit{Map} initialisiert. Die \textit{Map} ermöglicht eine Zuordnung zwischen den Spielern und ihren Scores. Im \textit{statistics}-Objekt werden die Statistiken (Averages und checkout Quote)pro Spieler gespeichert. Zuvor war es nur möglich averages (im \textit{playerAverages}-Objekt) zu speichern. Dieses Attribut kann in den nächsten Updates der Applikation gelöscht werden. Die \textit{legNumber} speichert die Nummer des Legs, welche beispielsweise für die Ausgabe in der Konsole (\textit{Leg Nummer 3 wird nun gespielt...}) verwendet. \textit{playerScoreAtRoundBegin} speichert den Score des Spielers zu Beginn einer Runde. Dieses Attribut wird benötigt, um die Checkout Quote zu berechnen. Das \textit{rounds}-Attribut speichert die verschiedenen Runden des Legs und das \textit{winner}-Attribut speichert den Gewinner eines Legs.
\subsection{Negativ Beispiel}
\begin{figure}[ht]
    \centering
    \includegraphics[width=0.8\textwidth]{Bilder/HandleLeg.png}
    \caption{HandleLeg Klasse}
    \label{fig:handleLeg-uml}
\end{figure}
Die HandleLeg-Klasse hält das SRP nicht vollständig ein, da sie eigentlich nur Logik für das Abarbeiten eines Legs beinhalten sollte. Die Logik des Legs befindet sich, wie auch bei den anderen Handle-Klassen, in der HandleLeg-Klasse, welche ein Leg-Objekt zurückgibt. Die Existenz der init-Methoden in der Klasse könnte auch schon als Argument genommen werden, um zu begründen, dass die Klasse das SRP nicht einhält, da diese Methoden auch in eine Klasse \textit{Setup} oder \textit{Init} ausgelagert werden könnten. Die beiden \textit{print}-Methoden verletzen das SRP jedoch deutlicher, da diese wenig mit dem Abarbeiten des Leg-Prozesses zu tun haben und ausgelagert werden können.
\section{Open Closed Principle (OCP)}

\subsection{Positiv Beispiel}
In der Leg-Klasse wird ein Statistics-Objekt verwendet, um verschiedene statistische Metriken zu speichern. Dies beinhaltet sowohl Durchschnittswerte, gespeichert in der \textit{averages}-Map, als auch Checkout-Informationen, die in der \textit{checkout}-Map gespeichert sind. Der Einsatz dieses Statistics-Objekts fördert die Einhaltung des Open-Closed-Prinzips (OCP).\\

Da das Statistics-Objekt für die Speicherung aller statistischen Werte verantwortlich ist, können neue Statistiken hinzugefügt oder vorhandene geändert werden, ohne die Leg-Klasse selbst zu modifizieren.\\

Sollte also eine Erweiterung um zusätzliche Statistiken erforderlich sein, würde dies lediglich eine Anpassung oder Erweiterung des Statistics-Objekts bedeuten. Die Leg-Klasse bleibt dadurch unverändert, wodurch das Risiko von Fehlern oder unerwarteten Seiteneffekten in diesem Teil des Systems minimiert wird. Dies unterstreicht, wie das Statistics-Objekt dazu beiträgt, das Open-Closed-Prinzip in der Anwendung einzuhalten.
\subsection{Negativ Beispiel}
Als negativ Beispiel können die verschiedenen Handle-Klassen betrachtet werden. Wenn beispielsweise ein weiterer Spielmodus hinzugefügt werden soll, muss jede Handle-Klasse angepasst werden. Beispielsweise ist das Kriterium für einen Checkout ein anderer bei bspw. Around the Clock als beim klassischen 501. Zum tracken der Punkte reicht auch keine normale Map mehr, sondern es wird eine separate Datenstruktur benötigt. Alle diese Änderungen würden dazu führen, dass die Handle-Klassen verändert werden müssten. Dies verstösst gegen das \acf{OCP}.
\section{\acf{ISP}}
\subsection{Positiv Beispiel}
\textbf{\acf{ISP}}\\
\begin{defStrich}[\acf{ISP}]
    Das \acf{ISP} ist ein Prinzip in der objektorientierten Programmierung, das besagt, dass keine Klasse aufgrund ihrer Schnittstellen von Methoden abhängig sein sollte, die sie nicht verwendet. Es gehört zu den fünf Prinzipien des SOLID-Ansatzes für das Software-Design. Mit anderen Worten, es ist besser, viele spezifische Schnittstellen zu haben als eine allgemeine; dies verhindert, dass eine Änderung in einer nicht verwendeten Methode einer Klasse zu einem unerwünschten Seiteneffekt in einer anderen Klasse führt. Das ISP hilft, den Code wartbarer und einfacher zu verstehen zu machen, indem es verhindert, dass Klassen mit ungenutzten Methoden überladen werden. Kurz gesagt, es fördert die Entkopplung von Software-Modulen und verbessert deren Flexibilität und Wiederverwendbarkeit.
\end{defStrich}
\begin{figure}[ht]
    \centering
    \includegraphics[width=0.8\textwidth]{Bilder/messagesDuringMatchUML.png}
    \caption{MessagesDuringMatch Interface UML}
    \label{fig:messagesDuringMatch-uml}
\end{figure}
Das Interface MessagesDuringMatch ist nicht monolithisch gestaltet, sondern in mehrere kleinere Interfaces unterteilt, die jeweils unterschiedliche Nachrichten während des Spielverlaufs repräsentieren. Diese Struktur entspricht dem Prinzip der \acf{ISP}. Im Falle, dass eine Klasse nur einen bestimmten Nachrichtentyp, wie zum Beispiel Statistikausgaben, benötigt, ermöglicht diese Struktur die Implementierung lediglich des relevanten Interfaces. Auf diese Weise wird eine effiziente und zielgerichtete Architektur erreicht.\newpage

\subsection{Negativ Beispiel}
\begin{figure}[ht]
    \centering
    \includegraphics[width=0.8\textwidth]{Bilder/HandleInterfaceUML.png}
    \caption{Handle Interface UML}
    \label{fig:handleinterface-uml}
\end{figure}
Das Handle-Interface verstößt gegen das \ac{ISP}, da es Methoden enthält, die nicht von allen Implementierungen benötigt werden. Konkret werden zwei Methoden bereitgestellt, wobei die erste Methode in allen Implementierungen genutzt wird, die zweite jedoch lediglich in zwei von sechs Implementierungen. Dies führt zu einer unnötigen Abhängigkeit für die vier Implementierungen, die die zweite Methode nicht benötigen. Eine Lösung dieses Problems könnte in der Aufteilung des Handle-Interfaces in zwei separate Interfaces bestehen, wobei jedes Interface genau eine der beiden Methoden enthält. Alternativ könnte das Interface auch vollständig aufgelöst und durch andere Mechanismen ersetzt werden, um den Anforderungen des \ac{ISP} gerecht zu werden.

\chapter{Weitere Prinzipien}
\section{Analyse GRASP: Geringe Kopplung}
\textbf{Geringe Kopplung} ist ein Prinzip von GRASP, das auf die Minimierung der Abhängigkeiten zwischen Klassen oder Modulen abzielt. Es ist eine grundlegende Strategie zur Steigerung der Modulierbarkeit und Wiederverwendbarkeit von Software. Bei geringer Kopplung sind einzelne Komponenten weniger auf die internen Eigenschaften und das Verhalten anderer Komponenten angewiesen. Dies erleichtert das Verständnis des Systems, da jede Komponente einzeln verstanden werden kann. Zudem wird dadurch die Wartbarkeit verbessert, da Änderungen an einer Komponente weniger wahrscheinlich Auswirkungen auf andere haben. Geringe Kopplung erleichtert auch das Testen von Komponenten, da diese unabhängig voneinander getestet werden können.
\subsection{Positiv Beispiel}
\begin{figure}[ht]
    \centering
    \includegraphics[width=0.7\linewidth]{Bilder/MatchHistoryUML.png}
    \caption{MatchHistory geringe Kopplung Positiv Beispiel}
    \label{fig:matchhistory-uml}
\end{figure}
Die Klasse \textit{MatchHistory} ist ein gutes Beispiel für geringe Kopplung in der Softwareentwicklung. Sie hat keine Abhängigkeiten zu anderen Klassen und kann somit unabhängig existieren und betrieben werden. Durch ihre Eigenständigkeit kann sie in jedem Moment des Programms ausgeführt oder ignoriert werden. Zum Beispiel kann der Benutzer am Ende des Programms entscheiden, ob die Match History gespeichert werden soll oder nicht. Ihre Funktionalität ist durch öffentliche Methoden wie \textit{getMatchHistoryString(Match)} und \textit{saveMatchHistory(Match, MessagesOutsideMatch)} sowie private Methoden definiert, die spezifische Aufgaben erfüllen. Alle diese Methoden sind speziell auf die Verarbeitung und Speicherung von Match-Daten zugeschnitten, und benötigen keine Kenntnisse oder Interaktionen mit anderen Teilen des Systems. Dies trägt dazu bei, die Kopplung zu minimieren und die Wartbarkeit und Erweiterbarkeit des Codes zu verbessern.

\subsection{Negativ Beispiel}
\begin{figure}[ht]
    \centering
    \includegraphics[width=0.7\linewidth]{Bilder/HandleThrowUML.png}
    \caption{HandleThrow geringe Kopplung Negativ Beispiel}
    \label{fig:handlethrow-uml}
\end{figure}
Die \textit{HandleThrow} Klasse zeigt einen deutlichen Mangel an geringer Kopplung. Sie ist stark an mehrere andere Klassen gebunden, einschließlich \textit{Player}, \textit{Leg}, \textit{MessagesDuringMatch}, \textit{Throw}, \textit{HandleDart}, \textit{Dart} und \textit{DartStatus}. Insbesondere führt die \textit{process()} Methode Änderungen am Zustand von \textit{Player}, \textit{Leg} und \textit{Dart}-Instanzen durch und ist somit stark an das Innenleben dieser Klassen gekoppelt. Änderungen an diesen Klassen könnten dazu führen, dass auch \textit{HandleThrow} angepasst werden muss. Außerdem verlässt sie sich auf die \textit{HandleDart} Klasse für das Verarbeiten eines Dart-Wurfs, was zusätzliche Abhängigkeiten schafft. Insgesamt fehlt dieser Klasse die Unabhängigkeit und Flexibilität, die durch geringe Kopplung erreicht werden könnte, was potenzielle Probleme bei der Wartung und Erweiterung des Codes verursachen könnte. 
Es könnte Abstraktion auf die \textit{HandleThrow}-Klasse eingesetzt werden. Dieses Prinzip bietet erhebliche Vorteile in Bezug auf Kapselung und Informationssicherheit. Anstatt direkt auf die Zustände der \textit{Player}, \textit{Leg} und \textit{Dart} Klassen zuzugreifen, könnte \textit{HandleThrow} durch Implementierung von Interfaces oder abstrakten Klassen mit diesen kommunizieren. 
\section{Analyse GRASP: Hohe Kohäsion}
\textbf{Hohe Kohäsion} bezieht sich auf das GRASP-Prinzip, das besagt, dass Klassen oder Module klar definierte, eng miteinander verbundene Verantwortlichkeiten haben sollten. Dieses Prinzip zielt darauf ab, dass die innerhalb einer Klasse oder eines Moduls gruppierten Funktionen stark miteinander verbunden sein sollten, um die Stabilität, Zuverlässigkeit und Verständlichkeit des Systems zu verbessern. Hohe Kohäsion erleichtert auch die Wartung und Erweiterung des Systems, da die Auswirkungen von Änderungen innerhalb einer Klasse auf andere Klassen minimiert werden. Sie fördert auch die Wiederverwendbarkeit von Klassen, da diese spezifische und gut definierte Aufgaben haben.\newline
\begin{figure}[ht]
    \centering
    \includegraphics[width=0.7\linewidth]{Bilder/UserCommunicationServiceUML.png}
    \caption{UserCommunicationService hohe Kohäsion Positiv Beispiel}
    \label{fig:UserCommunicationService-uml}
\end{figure}\\
Die Klasse \textit{UserCommunicationService} illustriert exemplarisch das Konzept der hohen Kohäsion in der objektorientierten Softwareentwicklung. Ihr alleiniger Zweck besteht in der Verarbeitung von Benutzereingaben, welche sie in ein entsprechendes \textit{UserInput}-Objekt umwandelt und zurückgibt. Zusätzlich initialisiert diese Klasse, falls nicht bereits geschehen, einen global verfügbaren \textit{Scanner} in einer dafür bestimmten Speicherklasse. Dieses Design wurde in Anbetracht zukünftiger Erweiterungen gewählt, bei denen weitere Klassen den Scanner nutzen könnten. Zudem optimiert es die aktuelle Version im Hinblick auf Testbarkeit, indem es die Simulation von Benutzereingaben mittels Mocking erleichtert.
\section{Don't Repeat Yourself (DRY)}
\textbf{\acf{DRY}} ist ein grundlegendes Prinzip in der Softwareentwicklung, das die Eliminierung von Redundanz fördert. Das Ziel ist, dass jede Information oder jedes Verhalten im System nur an einer Stelle definiert sein sollte. Dies verringert die Fehleranfälligkeit, da bei Änderungen oder Korrekturen nur eine Stelle im Code angepasst werden muss. Es verbessert auch die Wartbarkeit, da weniger Code zu warten ist und das System einfacher zu verstehen ist. DRY fördert zudem die Konsistenz im System, da die Wahrscheinlichkeit von Inkonsistenzen durch mehrfache Definitionen reduziert wird.\\\\
Das folgende Beispiel bezieht sich auf den Commit \textbf{\href{https://github.com/P3lina/ASE-Project/commit/270423b35cf9a364a2392e71870afeed7a87e809}{270423b}}\\
\begin{figure}[ht]
    \centering
    \includegraphics[width=0.85\linewidth]{Bilder/HandleDRYUML-vorher.png}
    \caption{HandleMatch und HandleSet DRY Positiv Beispiel (vorher)}
    \label{fig:handledry-vorher-uml}
\end{figure}\\
Wie in \autoref{fig:handledry-vorher-uml} zu sehen ist, enthalten die Klassen \textit{HandleMatch} und \textit{HandleSet} die nahezu gleichen Methoden \textit{isMatchSetWon} und \textit{getPlayerAndWinsOfPlayerWithMostSets}/\textit{LegsWon}. Diese beiden Methoden sind bei beiden Klassen nahezu identisch, womit knapp 50 Zeilen an Code-Duplikat existieren.\newpage
\begin{figure}[ht]
    \centering
    \includegraphics[width=1\linewidth]{Bilder/HandleDRYUML-nacher.png}
    \caption{HandleMatch und HandleSet DRY Positiv Beispiel (nacher)}
    \label{fig:handledry-nacher-uml}
\end{figure}
\autoref{fig:handledry-nacher-uml} zeigt, wie das Problem der doppelten Codepassagen gelöst wurde. Eine neue Klasse wurde implementiert, die als Elternklasse für die beiden vorherigen Klassen dient. Diese übergeordnete Klasse enthält die beiden Methoden, die vorher in beiden Klassen doppelt vorhanden waren.

Mit der Entfernung dieser doppelten Codepassagen lässt sich die Anzahl der Codezeilen um 50 Zeilen reduzieren. Dies trägt zur Lesbarkeit des gesamten Codes bei. Zudem zentralisiert diese Lösung die entsprechenden Methoden an einem Ort. Das erleichtert die Wartung, da bei Änderungen nur an einer Stelle Anpassungen vorgenommen werden müssen.
\chapter{Unit Tests}
\section{10 Unit Tests}

\newcounter{rowcounter}
\newcommand\rownumber{\stepcounter{rowcounter}\arabic{rowcounter}}

\newcolumntype{b}{X}
\newcolumntype{s}{>{\hsize=.5\hsize}X}

\begin{table}[H]
	\centering
	\begin{tabularx}{\textwidth}{rs|b}
		& \textbf{Unit Test} & \textbf{Beschreibung} \\
		\midrule
		\rownumber & processTest:\newline HandleDartTest &
        Der Test \textit{processTest()} überprüft, ob die \textit{process()}-Methode der Klasse HandleDart die Eingabe des Benutzers korrekt verarbeitet und die erwarteten Dartpunkte zurückgibt. Der Test wird fünfmal wiederholt, um mehrere Eingaben zu simulieren und die Konsistenz der Ergebnisse zu überprüfen. \\
        \rownumber & playerDartStatusTest:\newline HandleDartTest &
        Der Test \textit{playerDartStatusTest()} überprüft, ob die Methode \textit{getDartStatus()} der Klasse \textit{HandleDart} den korrekten Dartstatus basierend auf dem aktuellen Punktestand und dem geworfenen Dart zurückgibt. Der Test verwendet verschiedene Dart-Eingaben und überprüft, ob der zurückgegebene Dartstatus den erwarteten Werten entspricht. \\
        \rownumber & getPlayerAverageOf\newline RoundTest:Player\newline AverageCalculatorTest &
        Der Test \textit{getPlayerAverageOfRoundTest()} überprüft, ob die Methode \textit{getPlayerAverageOfRound()} des \textit{PlayerAverageCalculator} die durchschnittlichen Punkte eines Spielers in einer Runde korrekt berechnet und zurückgibt. \\
        \rownumber & getPlayerAverageOf\newline LegTest:PlayerAverage\newline CalculatorTest &
        Der Test \textit{getPlayerAverageOfLegTest()} überprüft, ob die Methode \textit{getPlayerAverageOfLeg()} des \textit{PlayerAverageCalculator} die durchschnittlichen Punkte eines Spielers über alle Runden in einem \textit{Leg} korrekt berechnet und zurückgibt. \\
	\end{tabularx}
	\caption{Unit Tests 1-4}
	\label{tab:tests}
\end{table}

\begin{table}[H]
	\centering
	\begin{tabularx}{\textwidth}{rs|b}
		& \textbf{Unit Test} & \textbf{Beschreibung} \\
		\midrule
        \rownumber & getPlayersAveragesOf\newline LegTest:Player\newline AverageCalculatorTest &
        Der Test \textit{getPlayersAveragesOfLegTest()} überprüft, ob die Methode \textit{getPlayersAveragesOfLeg()} des \textit{PlayerAverageCalculator} die durchschnittlichen Punkte aller Spieler in einem \textit{Leg} als \textit{Map} zurückgibt und ob die berechneten Durchschnittswerte den erwarteten Werten entsprechen. \\
        \rownumber & getPlayerCheckoutQuote\newline OfLegTest:PlayerCheck-outQuoteCalculatorTest &
        Der Test \textit{getPlayerCheckoutQuoteOfLegTest()} überprüft, ob die Methode \textit{getPlayerCheckoutQuoteOfLeg()} des \textit{PlayerCheckoutQuoteCalculator} den Checkout-Anteil (Prozentsatz des erfolgreichen Checkouts im Vergleich zu den möglichen Checkouts) eines bestimmten Spielers in einem Leg korrekt berechnet und zurückgibt. Der Test überprüft, ob der berechnete Anteil mit dem erwarteten Wert übereinstimmt. \\
        \rownumber & getPlayersCheckoutQuote\newline OfLegTest:PlayerCheck-outQuoteCalculatorTest &
        Der Test \textit{getPlayersCheckoutQuoteOfLegTest()} prüft, ob die Methode \textit{getPlayersCheckoutQuoteOfLeg()} des \textit{PlayerCheckoutQuoteCalculator} eine Map zurückgibt, die die Checkout-Prozentsätze aller Spieler in einem Leg enthält. Der Test überprüft, ob die berechneten Anteile für jeden Spieler mit den erwarteten Werten übereinstimmen. \\
        \rownumber & getUserInputTest:\newline UserCommunication\newline ServiceTest &
        Der Test \textit{getUserInputTest()} überprüft, ob die Methode \textit{getUserInput()} des \textit{UserCommunicationService} die Benutzereingabe korrekt abruft und als \textit{UserInput}-Objekt zurückgibt. Dabei wird eine vordefinierte Zeichenkette als simulierter Benutzereingabe verwendet, und der Test vergleicht, ob die Rückgabewerte der Methode für jede Zeile der simulierten Eingabe den erwarteten Werten entsprechen. \\
	\end{tabularx}
	\caption{Unit Tests 5-8}
	\label{tab:tests}
\end{table}

\begin{table}[H]
	\centering
	\begin{tabularx}{\textwidth}{rs|b}
		& \textbf{Unit Test} & \textbf{Beschreibung} \\
		\midrule
        \rownumber & isValidDartValid\newline CaseTest:UserInputTest &
        Der Test \textit{isValidDartValidCaseTest()} überprüft, ob die Methode \textit{isValidDart()} der Klasse \textit{UserInput} den Wert \textit{true} zurückgibt, wenn die Eingabe ein gültiger Dartwert ist. In diesem Fall wird überprüft, ob \textit{SBull} als gültiger Dartwert erkannt wird. \\
        \rownumber & prepareUserDartInput\newline SBullTest:UserInputTest &
        Der Test \textit{prepareUserDartInputSBullTest()} überprüft, ob die Methode \textit{prepareUserDartInput()} der Klasse \textit{UserInput} die Benutzereingabe \textit{25} korrekt in den Dartwert \textit{SBull} umwandelt und als \textit{UserInput}-Objekt zurückgibt. Dabei wird überprüft, ob das umgewandelte Objekt den erwarteten Wert \textit{SBull} hat. \\
	\end{tabularx}
	\caption{Unit Tests 9-10}
	\label{tab:tests}
\end{table}
\section{ATRIP: Automatic}
\begin{enumerate}
    \item Einfache Ausführung: Die Unit Tests werden ausgeführt und die Ergebnisse werden in der Konsole ausgegeben.
    \item Automatisches Ablaufen: Durch die Verwendung eines global zugänglichen Scanners, kann die Systemeingabe vom Test gesetzt werden und durch den öffentlichen Scanner immer wieder weitere Zeilen aufgerunfen werden.
    \item Selbstüberprüfung: Die Unit Tests liefern durch Asserts, wie AssertEquals oder AssertTrue immer nur das Ergebnis Bestanden / Nicht Bestanden.
\end{enumerate}\newpage
\section{ATRIP: Thorough}
\textbf{Positiv Beispiel:}\\
\begin{figure}[ht]
    \centering
    \includegraphics[width=0.8\textwidth]{Bilder/UserInputUML.png}
    \caption{UserInput UML}
    \label{fig:userinput-uml}
\end{figure}
In dieser Klasse wurden nur die Methoden \textit{isValidDart} und \textit{prepareUserDartInput} aktiv getestet. Es wurde sich dafür entschieden, da nur diese kritische und eventuell fehlerbehaftete Teile des Systems sind. Die Methoden \textit{toInt} und \textit{toString} sind hingegen kleine Hilfsmethoden, deren falsche Implementierung einen ebenso negativen Effekt auf die Anwendung hätte, jedoch sind diese Vergleichbar mit einer \textit{Getter} -und \textit{Setter}-Methode und sind somit trivial.\\
\lstinputlisting[
	label=code:userinputtest,    % Label; genutzt für Referenzen auf dieses Code-Beispiel
	caption=UserInputTest isValidDart-Methode,
	captionpos=b,               % Position, an der die Caption angezeigt wird t(op) oder b(ottom)
	style=EigenerJavaStyle,     % Eigener Style der vor dem Dokument festgelegt wurde
	firstline=1
]{Quellcode/UserInputTest.java}
Dieser Code Ausschnitt zeigt zwei Test-Methoden für die \textit{isValidDart}-Mathode. Dadurch werden beide Fälle der Methode abgedeckt: Ein gültiger Dart und ein ungültiger Dart.\\
\begin{figure}[ht]
    \centering
    \includegraphics[width=0.8\textwidth]{Bilder/thoroughexample.png}
    \caption{vollständig abgedeckte Methode \textit{isValidDart}}
    \label{fig:isValidDart-Coverage-Code}
\end{figure}\\
Wie auf diesem Bild zu sehen ist, wurde die Methode \textit{isValidDart} vollständig abgedeckt. Dies bedeutet, dass alle möglichen Pfade der Methode durchlaufen wurden.\\\\
\textbf{Negativ Beispiel:}\\
\begin{figure}[ht]
    \centering
    \includegraphics[width=0.8\textwidth]{Bilder/UserCommunicationServiceUML.png}
    \caption{UserCommunicationService UML}
    \label{fig:usercommunicationservice-uml}
\end{figure}
\lstinputlisting[
	label=code:usercommunicationservicetest,    % Label; genutzt für Referenzen auf dieses Code-Beispiel
	caption=UserCommunicationServiceTest Klasse,
	captionpos=b,               % Position, an der die Caption angezeigt wird t(op) oder b(ottom)
	style=EigenerJavaStyle,     % Eigener Style der vor dem Dokument festgelegt wurde
	firstline=1
]{Quellcode/UserCommunicationServiceTest.java}\newpage
\begin{figure}[ht]
    \centering
    \includegraphics[width=0.8\textwidth]{Bilder/UserCommunicationService-Thorough.png}
    \caption{UserCommunicationService abgedeckter Code}
    \label{fig:usercommunicationservice-abgedeckter-code}
\end{figure}
Dieses Beispiel bezieht sich auf die Klasse \textit{UserCommunicationService}, welche für die Benutzerinteraktion verantwortlich ist. Sie verfügt über eine Methode zur Erfassung der Benutzereingaben und zur Umwandlung dieser in ein \textit{UserInput}-Objekt. Obwohl der Code innerhalb des \textit{try}-Blocks normalerweise keine Probleme verursachen sollte, muss dennoch ein Verhalten der Anwendung definiert sein, das im Fehlerfall ausgeführt werden kann. Sollte es unerwartet zu einem Fehler kommen und dieses Verhalten ausgeführt werden, würde der Test fehlschlagen und das Verhalten der Anwendung wäre unvorhersehbar. Daher erfüllt dieser Test nicht das Kriterium Thorough aus dem ATRIP-Prinzip, da er einen wesentlichen Teil des Systems, auf dem die Anwendung basiert, nicht abdeckt.
\section{ATRIP: Professional}
\textbf{Positiv Beispiel:}\\
\lstinputlisting[
	label=code:create-leg,    % Label; genutzt für Referenzen auf dieses Code-Beispiel
	caption=\textit{createLeg}-Funktion der Unit Test Klasse \textit{PlayerAverageCalculatorTest},
	captionpos=b,               % Position, an der die Caption angezeigt wird t(op) oder b(ottom)
	style=EigenerJavaStyle,     % Eigener Style der vor dem Dokument festgelegt wurde
	firstline=1
]{Quellcode/createLeg_PlayerAverageCalculatorTest.java}
In der Unit Test Klasse \textit{PlayerAverageCalculatorTest} wurde die Methode \textit{createLeg} implementiert. Diese Methode wird in der \textit{setup()}-Methode aufgerufen. Durch die Auslagerung des Codes in diese Methode, kann der Code für zukünftige Tests wiederverwendet werden.\\
\lstinputlisting[
	label=code:playerListe,    % Label; genutzt für Referenzen auf dieses Code-Beispiel
	caption=\textit{players}-Liste der Unit Test Klasse \textit{PlayerAverageCalculatorTest},
	captionpos=b,               % Position, an der die Caption angezeigt wird t(op) oder b(ottom)
	style=EigenerJavaStyle,     % Eigener Style der vor dem Dokument festgelegt wurde
	firstline=1
]{Quellcode/playersListe_PlayerAverageCalculatorTest.java}
Zuvor wurden 2 \textit{Player}-Objekte initialisiert, doch um die Erweiterbarkeit der Klasse zu verbessern wurden diese 2 Objekte in eine Liste aus Spielern gespeichert. Dadurch kann in Zukunft die Anzahl der Spieler erhöht werden, ohne dass der Code der Unit Test Klasse angepasst werden muss.\newpage
\textbf{Negativ Beispiel:}\\
\begin{figure}[ht]
    \centering
    \includegraphics[width=0.8\textwidth]{Bilder/getUserInputTestCode.png}
    \caption{\textit{getUserInputTest}-Methode der Unit Test Klasse \textit{UserCommunicationServiceTest}}
    \label{fig:getUserInputTest-function}
\end{figure}\\
Als Negativ Beispiel könnte die Methode \textit{getUserInputTest} der Unit Test Klasse \textit{UserCommunicationServiceTest} dienen. Diese Methode ist für einen Entwickler, der nicht an der Entwicklung der Anwendung beteiligt war, nicht verständlich. Dies liegt daran, dass alle Operationen in einer Klasse geschehen. Durch die Einführung von Methoden, die den Code in kleinere Teile aufteilen, könnte die Lesbarkeit des Codes und die Wiederverwendbarkeit von Code-Teilen verbessert werden.\\
\section{Code Coverage}
Im Rahmen des Testens der Applikation wurde eine gezielte Vorgehensweise bei der Erstellung der Unit-Tests verfolgt. Es wurde dabei berücksichtigt, dass nicht für jede Klasse Tests notwendig sind, sondern ein Fokus auf jene gelegt wurde, die für den Verlauf der Anwendung als relevant erachtet wurden.

Die Konzentration lag insbesondere auf Klassen oder Methoden, die durch ihre Komplexität oder Wichtigkeit in der Anwendung hervorstechen. Ein Beispiel hierfür ist die Berechnung des Player-Average, welche eine komplexere Berechnung in vielen Teilen darstellt und daher explizit getestet wurde.

Aus diesem Grund wurden ausschließlich Tests für die Application-Layer entwickelt, da sie eine entscheidende Rolle für den Verlauf der Anwendung spielt. Die in dieser Schicht enthaltenen Klassen und Methoden haben direkte Auswirkungen auf den Anwendungsverlauf, weshalb ihre korrekte Funktion von großer Bedeutung ist.

Trotz alleinigen Testens der Application-Layer, blieb die Domain-Layer nicht unbeachtet. Viele der Klassen und Methoden dieser Schicht wurden passiv durch die Unit-Tests in der Application-Layer abgedeckt. Dies hat dazu geführt, dass die Domain-Layer eine Line-Coverage von knapp 50\% erreicht hat.

Zusammengefasst lässt sich sagen, dass durch diese Vorgehensweise eine gute und effiziente Testabdeckung erreicht wurde, die einen guten Überblick über den Zustand und die Qualität der Anwendung gibt.\newpage
\begin{figure}[ht]
    \centering
    \includegraphics[width=0.7\textwidth]{Bilder/testCoverage.png}
    \caption{Überblick über die Test Coverage}
    \label{fig:test-coverage}
\end{figure}
\autoref{fig:test-coverage} gibt einen Überblick über die Abdeckung der implementierten Tests.\newpage
\section{Fakes und Mocks}
\begin{figure}[ht]
    \centering
    \includegraphics[width=0.8\textwidth]{Bilder/MockOverview.png}
    \caption{Übersicht der Mock-Klassen}
    \label{fig:mock}
\end{figure}
In der Unit Test Klasse \textit{MatchHistoryTest} werden zwei Fake- bzw. Mock-Objekte verwendet, um die Funktionalität der \textit{getMatchHistoryString} Methode in der MatchHistory Klasse zu testen. Die zwei Mock-Objekte, \textit{PlayerMock} und \textit{MatchMock}, simulieren das Verhalten realer Objekte innerhalb eines isolierten Testumfelds.

\textit{PlayerMock} dient dazu, das Verhalten eines \textit{Player}-Objekts zu imitieren, insbesondere im Hinblick auf die Rückgabe eines spezifischen Spielernamens. Auf der anderen Seite stellt \textit{MatchMock} ein komplexeres Szenario dar, in dem ein vollständiges Dartspiel mit Sets, Legs, Runden und Würfen simuliert wird.

Der Einsatz dieser Mock-Objekte ist wichtig, da es das Testen der getMatchHistoryString Methode ermöglicht, ohne von den tatsächlichen Implementierungen der Player und Match Klassen abhängig zu sein. Dies ist besonders nützlich, da die realen Klassen, vor allem die Match-Klasse, komplex sind.

Diese Mock-Objekte erlauben das Erstellen von vorhersehbaren Testszenarien. In diesem Fall kann garantiert werden, dass der \textit{getMatchHistoryString} Methode immer das gleiche Match mit den gleichen Spielern übergeben wird, wodurch der erwartete Ausgabestring konstant bleibt.
\chapter{Domain Driven Design}
\section{Ubiquitous Language}
Es wurde sich bereits vor der Implementierung des Projektes gedanken über die Ubiquitous Language gemacht, wie in dem Commit \href{https://github.com/P3lina/ASE-Project/commit/0005472ee771b6231a118706708e1d19900958c2}{\textbf{0005472}} zu sehen ist. Für diese Vorgehensweise wurde sich bewusst entschieden, da die Namensgebung aller Klassen, Objekte, Variablen, Nachrichten usw. auf den Namen der Ubiquitous Language basieren.\\
Beispiele der Ubiquitous Language können auf der nächsten Seite gesehen werden.\\
\newcounter{ubctr}
\newcommand\unr{\stepcounter{ubctr}\arabic{ubctr}}

\newcolumntype{b}{X}
\newcolumntype{s}{>{\hsize=.5\hsize}X}

\begin{table}[H]
	\centering
	\begin{tabularx}{\textwidth}{rs|bb}
		& \textbf{Bezeichnung} & \textbf{Bedeutung} & \textbf{Begründung} \\
		\midrule
		\unr & \textbf{Match} & Ein Match ist das gesamte Spiel zwischen mehreren Spielern. Ein Spiel besteht aus mindestens einem Set. Ein Set besteht aus mindestens einem Leg. Ein Leg besteht aus mindestens einer Runde. & Vor allem für Dart-Neulingen könnten Dart Begriffe nicht geläufig sein und daher könnte ein Neuling davon ausgehen, dass ein Match gleich eneis Legs ist.\\
        \unr & \textbf{Score} & Der Score ist der Punktestand eines Spielers in einem Leg. Im klassischen Dart liegt er somit zwischen 0 und 501 Punkten. & Für den Score muss auch ein bestimmtes Wort festgelegt werden, damit die Verwendung von verschiedenen Begriffen, wie z.B. Points nicht zu Verwirrungen führen.\\
        \unr & \textbf{Dart} & Ein Dart ist ein Wurf auf die Dartscheibe. Ein Dart besteht aus einem Wert des Darts (\zB 60) und der Information, ob dieser Dart ein Doppelfeld getroffen hat, da dies wichtig ist für das Beenden eines Legs. & Es musste ein Begriff für einen einzelnen Dart festgelegt werden, damit er nicht mit dem Throw verwechselt wird.\\
        \unr & \textbf{Throw} & Ein Throw besteht aus drei geworfenen Darts auf die Dartscheibe. & Dieser Begriff wurde eingeführt, damit er nicht mit dem Einzelnen Dart-Wurf (Dart) verwechselt wird.\\
	\end{tabularx}
	\caption{Ubiquitous Language}
	\label{tab:ublang}
\end{table}
\section{Entities}
Eine Entity sei definiert über Identität, Lebenszyklus und Verhalten.\\
\textbf{Entity Match}:\\
\begin{figure}[ht]
    \centering
    \includegraphics[width=0.4\textwidth]{Bilder/match.png}
    \caption{Match UML}
    \label{fig:match-uml}
\end{figure}
\begin{itemize}
    \item \textbf{Identität:} Jedes Match ist eindeutig und kann anhand einer Identität unterschieden werden. Auch wenn die Attribute (Spieler, Sets, etc.) des Matches sich ändern könnten, bleibt die Identität des Matches gleich. Die Identität könnte implizit durch eine Kombination der Attribute oder explizit durch eine eindeutige ID repräsentiert werden, falls nötig.
    \item Ein Match hat einen klaren Lebenszyklus und einen Zustand, der sich im Verlauf der Applikation ändert. Es wird initialisiert, Sets werden hinzugefügt, und schließlich wird ein Gewinner festgelegt.
    \item Das Match hat Methoden, die das Verhalten darstellen, wie ein Match manipuliert oder geändert werden kann.
\end{itemize}
\section{Value Objects}
In dieser Anwendung wurden keine vollwertigen Value Objects verwendet, da der Einsatz dieser nicht notwendig ist. Value Objects werden oft für elementare Werte verwendet, wie Geld oder eine Adresse, wobei solche Konzepte in der Anwendung nicht vorkommen.\\
Jedoch könnte ein abgewandeltes Objekt der Dart-Klasse als Value Object angesehen werden, da ein Dart Wurf insbesondere über sein Attribut \textit{points} definiert wird und somit zwei Dart Objekte mit der gleichen points-Anzahl als gleich angesehen werden können.
\section{Repositories}
Repositories sind besonders nützlich, wenn wiederholte Zugriffe auf bestimmte Objekte erforderlich sind, da sie das Definieren von spezifischem Suchverhalten für diese Objekte ermöglichen.\\
In dieser Anwendung wird immer nur am Ende eines Legs oder Sets auf andere Objekte zugegriffen. Zum Beispiel wird am Ende eines Legs auf die vorherigen Legs zugegriffen, um einen eventuellen Gewinner des Sets zu ermitteln. Da diese Zugriffe so selten geschehen, ist der Einsatz eines Repositories überflüssig und würde nur unnötig die Komplexität erhöhen.
\section{Aggregates}

Aggregates in \acf{DDD} sind Gruppen von assoziierten Objekten, die als Einheit behandelt werden, um die Konsistenz und Integrität der Geschäftsdaten zu gewährleisten. Jedes Aggregate hat eine sogenannte Root-Entität, die als Einstiegspunkt für die Interaktion mit dem Aggregate dient und die Kontrolle über den Lebenszyklus der inneren Objekte hat. Zugriffe auf innere Objekte eines Aggregats erfolgen in der Regel über die Root-Entität. So wird sichergestellt, dass Geschäftsregeln eingehalten werden und die Datenkonsistenz gewährleistet ist. Aggregates helfen, die Komplexität im Design zu reduzieren und die Isolation zwischen verschiedenen Teilen des Systems zu fördern. In diesem Projekt wurde kein spezifisches Aggregate implementiert, da der Entwurf so gewählt wurde, dass jedes Objekt für seine eigene Konsistenz zuständig ist.
\chapter{Refactoring}
\section{Code Smells}
\textbf{Code Smell 1 (Long Method):}\\
\lstinputlisting[
	label=code:vorRef,    % Label; genutzt für Referenzen auf dieses Code-Beispiel
	caption=getPlayerAverageOfLeg-Methode vor Refactoring,
	captionpos=b,               % Position, an der die Caption angezeigt wird t(op) oder b(ottom)
	style=EigenerJavaStyle,     % Eigener Style der vor dem Dokument festgelegt wurde
	firstline=1,                
]{Quellcode/getPlayerAverageOfLeg.java}
\lstinputlisting[
	label=code:nachref,    % Label; genutzt für Referenzen auf dieses Code-Beispiel
	caption=getPlayerAverageOfLeg-Methode nach Refactoring,
	captionpos=b,               % Position, an der die Caption angezeigt wird t(op) oder b(ottom)
	style=EigenerJavaStyle,     % Eigener Style der vor dem Dokument festgelegt wurde
	firstline=1,                
]{Quellcode/getPlayerAverageOfLeg_refactored.java}
Die Methode \textit{getPlayerAverageOfLeg} wurde so überarbeitet, dass sie klarer und einfacher zu verstehen ist. Dies wurde durch die Extraktion von Teilen des Codes in separate, private Hilfsmethoden erreicht, die jeweils eine spezifische Aufgabe erfüllen.

Die Methode \textbf{getRoundNumberOfLastRound}: Diese Methode nimmt als Eingabe ein Leg und gibt die Rundennummer der letzten Runde zurück. Sie kapselt den Teil des ursprünglichen Codes, der die Gesamtzahl der Runden ermittelt.

Die Methode \textbf{getSumOfPlayerAveragesOfRounds}: Diese Methode nimmt als Eingabe eine Liste von Round-Objekten und einen Player und gibt die Summe der Durchschnittswerte des Spielers pro Runde zurück. Sie kapselt den Teil des ursprünglichen Codes, der durch die Runden iteriert, um den Durchschnitt des Spielers für jede Runde zu berechnen.

Die Hauptmethode getPlayerAverageOfLeg wurde nun so vereinfacht, dass sie leichter zu verstehen ist. Sie ruft nun einfach die Hilfsmethoden auf und führt eine Berechnung durch, um den Durchschnittswert des Spielers für das Leg zu ermitteln. Dies bietet die Vorteile Wartbarkeit, Lesbarkeit, Testbarkeit und Wiedrverwendbarkeit.\\\\
\textbf{Code Smell 2 (Duplicated Code):}\\
Die Klassen \textit{HandleMatch} und \textit{HandleSet} weisen beide eine Methode auf, die prüft, ob ein Spieler entweder ein komplettes Spiel oder ein Set gewonnen hat. Diese Methoden sind in ihrer Struktur und Logik sehr ähnlich. Der einzige Unterschied besteht darin, dass die Methode in HandleMatch die unterschiedlichen Sets und deren Gewinner berücksichtigt, während die Methode in HandleSet die verschiedenen Legs eines Sets und deren Gewinner in Betracht zieht. Ansonsten sind beide Methoden nahezu identisch. Dies stellt einen deutlichen Fall von doppeltem Code dar, durch dessen Vermeidung etwa 50 Zeilen Code eingespart und die Lesbarkeit, Wiederverwendbarkeit sowie Testbarkeit des Codes verbessert werden könnten.

Zur Lösung dieses Problems mit doppeltem Code wurde die Klasse \textit{HandleGame} eingeführt. Diese Klasse enthält die duplizierten Methoden und dient als Elternklasse für HandleMatch und HandleSet.
\lstinputlisting[
	label=code:ismatchsetwon,    % Label; genutzt für Referenzen auf dieses Code-Beispiel
	caption=Methode isMatchSetWon der Elternklasse,
	captionpos=b,               % Position, an der die Caption angezeigt wird t(op) oder b(ottom)
	style=EigenerJavaStyle,     % Eigener Style der vor dem Dokument festgelegt wurde
	firstline=1,                
]{Quellcode/isMatchSetWon.java}
Die Methode in \autoref{code:ismatchsetwon} wurde wie bereits erwähnt in die Elternklasse HandleGame ausgelagert, womit dieser duplizierte Code gespart wurde.
\lstinputlisting[
	label=code:ismatchsetwonhandleset,    % Label; genutzt für Referenzen auf dieses Code-Beispiel
	caption=Methode isMatchSetWon der HandleSet,
	captionpos=b,               % Position, an der die Caption angezeigt wird t(op) oder b(ottom)
	style=EigenerJavaStyle,     % Eigener Style der vor dem Dokument festgelegt wurde
	firstline=1,                
]{Quellcode/isMatchSetWon_HandleSet.java}
Die Methode in \autoref{code:ismatchsetwonhandleset} ist nun die neue Methode in HandleSet. Die Klasse HandleSet erbt von der Klasse HandleGame, womit die gleiche Funktionalität, wie vor dem Refactoring erreicht wurde.\newpage
\section{2 Refactorings}
\textbf{Refactoring 1 (Extract Method):}\\
Dieses Beispiel bezieht sich auf Commit \href{https://github.com/P3lina/ASE-Project/commit/f50cfec78b045a3a62e5927e44d2699f528aee31}{f50cfec}.\\
\begin{figure}[ht]
    \centering
    \includegraphics[width=0.6\textwidth]{Bilder/exmeth.png}
    \caption{Extract Method in der \textit{fillPlayerScoreAtRoundBeginMap}-Methode}
    \label{fig:exmeth}
\end{figure}\\
Wie bereits in \autoref{fig:exmeth} in Teilen zu sehen ist, ist die vorgestellte Methode sehr lang und damit unübersichtlich, weswegen sich entschieden wurde, diese Methode in mehrere kleinere, übersichtlichere Methode aufzuteilen.\\
\begin{figure}[ht]
    \centering
    \includegraphics[width=0.6\textwidth]{Bilder/ref1before.png}
    \caption{UML vor Refactor}
    \label{fig:ref1before}
\end{figure}\newpage
\begin{figure}[ht]
    \centering
    \includegraphics[width=0.6\textwidth]{Bilder/ref1after.png}
    \caption{UML nach Refactor}
    \label{fig:ref1after}
\end{figure}
Wie in \autoref{fig:ref1before} und \autoref{fig:ref1after} zu sehen, wurden 2 neue Methoden hinzugefügt, welche jeweils Verantwortung für einen Teil der ursprünglichen Methode übernehmen. Die Methode \textit{fillPlayerScoreAtRoundBeginMap} ist nun übersichtlicher und leichter zu verstehen.\\\\
\textbf{Refactoring 2 (Replace Conditional with Polymorphism):}\\
Dieses Beispiel bezieht sich auf Commit \href{https://github.com/P3lina/ASE-Project/commit/479ad5e6fddd23eb7df3697e75d55b52b255b301}{479ad5e}\\
\lstinputlisting[
	label=code:prepare,    % Label; genutzt für Referenzen auf dieses Code-Beispiel
	caption=Code vor Refactor,
	captionpos=b,               % Position, an der die Caption angezeigt wird t(op) oder b(ottom)
	style=EigenerJavaStyle,     % Eigener Style der vor dem Dokument festgelegt wurde
	firstline=1,
]{Quellcode/prepareUserDartInput.java}
Wie in \autoref{code:prepare} zu sehen ist, verwendet die Methode 5 if-Statements und einen Default Fall. Dieses if-Statement könnte im Verlauf der Applikation erweitert werden oder es könnten Statements gelöscht oder erweitert werden. Eine hohe Anzahl von if-Zweigen erhöht die Komplexität einer Methode, was das Lesen, Verstehen und Pflegen des Codes erschwert.

Wenn neue Fälle berücksichtigt werden sollen, erfordert dies zusätzliche Arbeit und birgt das Risiko, Fehler zu erzeugen. Zudem kann die Einfügung neuer Bedingungen an irgendeiner Stelle im if-Konstrukt unerwartete Auswirkungen auf die restliche Logik haben.

Zusätzlich kann die Testbarkeit der Methode leiden, da jeder if-Zweig individuell getestet werden muss. Dies erhöht die Anzahl der benötigten Testfälle und die Zeit, die für das Schreiben und Ausführen der Tests benötigt wird.
Deswegen wurde sich entschieden, dieses if-Statement durch Polymorphismus zu ersetzen.\\
\begin{figure}[ht]
    \centering
    \includegraphics[width=0.9\textwidth]{Bilder/rcwp.png}
    \caption{UML nach Refactor}
    \label{fig:rcwp}
\end{figure}\\
Wie nun in \autoref{fig:rcwp} zu sehen ist, wurden 6 neue Klassen eingeführt. \textit{UserInputCase} ist dabei das Interface, auf dem die anderen Fälle aufbauen und welches die benötigten Methoden für Zutrefflichkeit und Konsens bereitstellt. Diese 5 Klassen implementieren das Interface und überschreiben die Methoden. Die Klasse \textit{UserInputCaseDefault} ist dabei beispielsweise der Default Fall, welcher die Logik des Default Falls aus der ursprünglichen Methode übernimmt.
\chapter{Entwurfsmuster}
\section{Entwurfsmuster: }
\section{Entwurfsmuster: }

% ---- Literaturverzeichnis
\cleardoublepage
\renewcommand*{\chapterpagestyle}{plain}
\pagestyle{plain}
\pagenumbering{Roman}                   % Römische Seitenzahlen
\setcounter{page}{\numexpr\value{savepage}+1}
\printbibliography[title=Literaturverzeichnis]

% ---- Anhang
\appendix
%\clearpage
%\pagenumbering{Roman}  % römische Seitenzahlen für Anhang

\newpage
\end{document}
