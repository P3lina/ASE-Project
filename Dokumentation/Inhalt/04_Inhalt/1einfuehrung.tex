\chapter{Einführung}
\section{Übersicht über die Applikation}
Dart ist ein beliebtes Geschicklichkeitsspiel, das sowohl als professioneller Sport als auch als geselliges Freizeitspiel gespielt wird. Ziel des Spiels ist es, Punkte zu sammeln, indem man Pfeile (\textit{Darts}) auf ein kreisförmiges Dartboard wirft. Ein Standard-Dartboard ist in 20 nummerierte Sektionen unterteilt, wobei jeder Bereich unterschiedliche Punktzahlen vergibt. Darüber hinaus gibt es Doppel (Double) und Dreifach (Triple Felder), die den Wert der getroffenen Sektion verdoppeln bzw. verdreifachen. Das Feld dreifache 20 (\textit{triple 20}) gibt die meisten Punkte. Es gibt verschiedene Spielvarianten, wobei 501 und 301 die bekanntesten sind. Bei diesen Varianten beginnen die Spieler mit einer festgelegten Punktzahl und müssen versuchen, ihre Punktzahl exakt auf Null zu reduzieren. Das Spiel erfordert Präzision, gute Hand-Auge-Koordination und strategisches Denken. Professionelle Dart-Turniere werden weltweit ausgetragen und ziehen Tausende von Zuschauern an. Der Dart-Counter ist eine Anwendung, die es einer Gruppe aus Spielern ermöglicht dieses Spiel zu spielen. Zwar müssen die Spieler die Darts immer noch händisch auf eine Dartscheibe werfen, jedoch können die Spieler so ihre Punkte zählen und wissen, wie viele Punkte ihnen noch zu einem Checkout fehlen. Darüber hinaus können die Spieler einen ständigen Überblick über ihren aktuellen Average (Durchschnitt) behalten und am Ende eines Legs (\zB einer Runde 501) ihre Checkout-Quote (Trefferquote) sehen. Am Ende eines Gesamten Matches, welches aus mehreren Sets bestehen, welche wiederum aus mehreren Legs bestehen, haben die Spieler die Möglichkeit, ihren Spielverlauf in einer Textdatei zu speichern.
\section{Wie startet man die Applikation}
Zum starten der Applikation wird eine Java-Laufzeitumgebung und ein JDK (Version 19) benötigt. Zusätzlich wird eine \acf{IDE} zum Ausführen der Anwendung benötigt. Mit dieser IDE kann auch die Anwendung gebaut werden, um das Ausführen ohne IDE zu ermöglichen. Gestartet werden muss die Klasse '\textit{DartCounterV2/src/main/java/de/p3lina/Main.java}'. Alle Interaktionen mit der Anwendung finden dann in dem Terminal der IDE statt, mit welcher ausschlieslich mit der Tastatur interagiert werden kann. Um ein eigentliches Match zu spielen, werden zuerst Spielinformationen, wie \zB Anzahl der Spieler, Spielernamen, Startpunktanzahl etc. benötigt. Diese können, wie bereits erwähnt mit der Tastatur spezifiziert werden.
\section{Wie testet man die Applikation}
Das Testen der Applikation erfordert die Installation von Maven. Zum Testen können die Test-Klassen, welche sich in jedem Modul (application, domain etc.) unter '\textit{src/main/test}' befinden, ausgeführt werden. Die Test-Klassen können mit der IDE ausgeführt werden. Um nicht alle Tests einzeln ausführen zu müssen, besteht auch die Möglichkeit mit dem Terminal in das Wurzelverzeichnis zu navigieren und den Befehl '\textit{mvn test}' auszuführen.